%% abtex2-modelo-trabalho-academico.tex, v-1.9.7 laurocesar
%% Copyright 2012-2018 by abnTeX2 group at http://www.abntex.net.br/ 
%%
%% This work may be distributed and/or modified under the
%% conditions of the LaTeX Project Public License, either version 1.3
%% of this license or (at your option) any later version.
%% The latest version of this license is in
%%   http://www.latex-project.org/lppl.txt
%% and version 1.3 or later is part of all distributions of LaTeX
%% version 2005/12/01 or later.
%%
%% This work has the LPPL maintenance status `maintained'.
%% 
%% The Current Maintainer of this work is the abnTeX2 team, led
%% by Lauro César Araujo. Further information are available on 
%% http://www.abntex.net.br/
%%
%% This work consists of the files abntex2-modelo-trabalho-academico.tex,
%% abntex2-modelo-include-comandos and abntex2-modelo-references.bib
%%

% ------------------------------------------------------------------------
% ------------------------------------------------------------------------
% abnTeX2: Modelo de Trabalho Academico (tese de doutorado, dissertacao de
% mestrado e trabalhos monograficos em geral) em conformidade com 
% ABNT NBR 14724:2011: Informacao e documentacao - Trabalhos academicos -
% Apresentacao
% ------------------------------------------------------------------------
% ------------------------------------------------------------------------

\documentclass[
	% -- opções da classe memoir --
	12pt,				% tamanho da fonte
	openright,			% capítulos começam em pág ímpar (insere página vazia caso preciso)
	twoside,			% para impressão em recto e verso. Oposto a oneside
	a4paper,			% tamanho do papel. 
	% -- opções da classe abntex2 --
	%chapter=TITLE,		% títulos de capítulos convertidos em letras maiúsculas
	%section=TITLE,		% títulos de seções convertidos em letras maiúsculas
	%subsection=TITLE,	% títulos de subseções convertidos em letras maiúsculas
	%subsubsection=TITLE,% títulos de subsubseções convertidos em letras maiúsculas
	% -- opções do pacote babel --
	english,			% idioma adicional para hifenização
	french,				% idioma adicional para hifenização
	spanish,			% idioma adicional para hifenização
	brazil				% o último idioma é o principal do documento
	]{abntex2}

% ---
% Pacotes básicos 
% ---
\usepackage{lmodern}			% Usa a fonte Latin Modern			
\usepackage[T1]{fontenc}		% Selecao de codigos de fonte.
\usepackage[utf8]{inputenc}		% Codificacao do documento (conversão automática dos acentos)
\usepackage{indentfirst}		% Indenta o primeiro parágrafo de cada seção.
\usepackage{color}				% Controle das cores
\usepackage{graphicx}			% Inclusão de gráficos
\usepackage{microtype} 			% para melhorias de justificação
% ---
		
% ---
% Pacotes adicionais, usados apenas no âmbito do Modelo Canônico do abnteX2
% ---
\usepackage{lipsum}				% para geração de dummy text
% ---

% ---
% Pacotes de citações
% ---
\usepackage[brazilian,hyperpageref]{backref}	 % Paginas com as citações na bibl
\usepackage[alf]{abntex2cite}	% Citações padrão ABNT

% --- 
% CONFIGURAÇÕES DE PACOTES
% --- 

% ---
% Configurações do pacote backref
% Usado sem a opção hyperpageref de backref
\renewcommand{\backrefpagesname}{Citado na(s) página(s):~}
% Texto padrão antes do número das páginas
\renewcommand{\backref}{}
% Define os textos da citação
\renewcommand*{\backrefalt}[4]{
	\ifcase #1 %
		Nenhuma citação no texto.%
	\or
		Citado na página #2.%
	\else
		Citado #1 vezes nas páginas #2.%
	\fi}%
% ---

% ---
% Informações de dados para CAPA e FOLHA DE ROSTO
% ---
\titulo{PIM V \\ Projeto Integrado Multidiciplinar V}
\autor{Pedro José Laurenti de Matos
\\Fernando de Sampaio Steimback Vidal
\\Larissa Araújo
\\Cleverson Souza
\\Luiz Gustavo
} %falta%
\local{Brasil}
\data{Março de 2023}
\orientador{} %falta%
\coorientador{} %falta%
\instituicao{
  Universidade Paulista -- UNIP
  \par
  Curso de Análise e Desenvolvimento de Sistemas
  \par
  Programa de Pós-Graduação}
\tipotrabalho{Trabalho Científico}
% O preambulo deve conter o tipo do trabalho, o objetivo, 
% o nome da instituição e a área de concentração 
\preambulo{Trabalho científico produzido para colocar em prática as habilidades adquiridas no segundo período do curso de Análise e Desenvolvimento de Sistemas.}
% ---


% ---
% Configurações de aparência do PDF final

% alterando o aspecto da cor azul
\definecolor{blue}{RGB}{41,5,195}

% informações do PDF
\makeatletter
\hypersetup{
     	%pagebackref=true,
		pdftitle={\@title}, 
		pdfauthor={\@author},
    	pdfsubject={\imprimirpreambulo},
	    pdfcreator={LaTeX with abnTeX2},
		pdfkeywords={abnt}{latex}{abntex}{abntex2}{trabalho acadêmico}, 
		colorlinks=true,       		% false: boxed links; true: colored links
    	linkcolor=blue,          	% color of internal links
    	citecolor=blue,        		% color of links to bibliography
    	filecolor=magenta,      		% color of file links
		urlcolor=blue,
		bookmarksdepth=4
}
\makeatother
% --- 

% ---
% Posiciona figuras e tabelas no topo da página quando adicionadas sozinhas
% em um página em branco. Ver https://github.com/abntex/abntex2/issues/170
\makeatletter
\setlength{\@fptop}{5pt} % Set distance from top of page to first float
\makeatother
% ---

% ---
% Possibilita criação de Quadros e Lista de quadros.
% Ver https://github.com/abntex/abntex2/issues/176
%
\newcommand{\quadroname}{Quadro}
\newcommand{\listofquadrosname}{Lista de quadros}

\newfloat[chapter]{quadro}{loq}{\quadroname}
\newlistof{listofquadros}{loq}{\listofquadrosname}
\newlistentry{quadro}{loq}{0}

% configurações para atender às regras da ABNT
\setfloatadjustment{quadro}{\centering}
\counterwithout{quadro}{chapter}
\renewcommand{\cftquadroname}{\quadroname\space}
\renewcommand*{\cftquadroaftersnum}{\hfill--\hfill}

\setfloatlocations{quadro}{hbtp} % Ver https://github.com/abntex/abntex2/issues/176
% ---

% --- 
% Espaçamentos entre linhas e parágrafos 
% --- 

% O tamanho do parágrafo é dado por:
\setlength{\parindent}{1.3cm}

% Controle do espaçamento entre um parágrafo e outro:
\setlength{\parskip}{0.2cm}  % tente também \onelineskip

% ---
% compila o indice
% ---
\makeindex
% ---

% ----
% Início do documento
% ----
\begin{document}

% Seleciona o idioma do documento (conforme pacotes do babel)
%\selectlanguage{english}
\selectlanguage{brazil}

% Retira espaço extra obsoleto entre as frases.
\frenchspacing 

% ----------------------------------------------------------
% ELEMENTOS PRÉ-TEXTUAIS
% ----------------------------------------------------------
% \pretextual

% ---
% Capa
% ---
\imprimircapa
% ---

% ---
% Folha de rosto
% (o * indica que haverá a ficha bibliográfica)
% ---
\imprimirfolhaderosto*
% ---

% ---
% Inserir a ficha bibliografica
% ---

% Isto é um exemplo de Ficha Catalográfica, ou ``Dados internacionais de
% catalogação-na-publicação''. Você pode utilizar este modelo como referência. 
% Porém, provavelmente a biblioteca da sua universidade lhe fornecerá um PDF
% com a ficha catalográfica definitiva após a defesa do trabalho. Quando estiver
% com o documento, salve-o como PDF no diretório do seu projeto e substitua todo
% o conteúdo de implementação deste arquivo pelo comando abaixo:
%
% \begin{fichacatalografica}
%     \includepdf{fig_ficha_catalografica.pdf}
% \end{fichacatalografica}

\begin{fichacatalografica}
	\sffamily
	\vspace*{\fill}					% Posição vertical
	\begin{center}					% Minipage Centralizado
	\fbox{\begin{minipage}[c][8cm]{13.5cm}		% Largura
	\small
	\imprimirautor
	%Sobrenome, Nome do autor
	
	\hspace{0.5cm} PIM V Projeto Integrado Multidiciplinar V --
	\imprimirlocal, \imprimirdata \ --
	
	\hspace{0.5cm} \thelastpage p. : il. (algumas color.) ; 30 cm.\\
	
	\hspace{0.5cm} \imprimirorientadorRotulo~\imprimirorientador\\
	
	\hspace{0.5cm}
	\parbox[t]{\textwidth}{\imprimirtipotrabalho~--~\imprimirinstituicao,
	\imprimirdata.}\\
	
	\hspace{0.5cm}
		1. Sistemas educacionais. \ \
		2. Desenvolvimento de Software. \ \
		3. Análise e Desenvolvimento de Sistemas. \ \
		4. ABNT LaTex \ \
		5. Universidade UNIP. \ \
		6. Faculdade de xxx. \ \
		7. Projeto Integrado Multidiciplinar V. \ \		
	\end{minipage}}
	\end{center}
\end{fichacatalografica}
% ---

% ---

% ---
% Inserir folha de aprovação
% ---

% Isto é um exemplo de Folha de aprovação, elemento obrigatório da NBR
% 14724/2011 (seção 4.2.1.3). Você pode utilizar este modelo até a aprovação
% do trabalho. Após isso, substitua todo o conteúdo deste arquivo por uma
% imagem da página assinada pela banca com o comando abaixo:
%
% \begin{folhadeaprovacao}
% \includepdf{folhadeaprovacao_final.pdf}
% \end{folhadeaprovacao}
%
\begin{folhadeaprovacao}

  \begin{center}
    {\ABNTEXchapterfont\large\imprimirautor}

    \vspace*{\fill}\vspace*{\fill}
    \begin{center}
      \ABNTEXchapterfont\bfseries\Large\imprimirtitulo
    \end{center}
    \vspace*{\fill}
    
    \hspace{.45\textwidth}
    \begin{minipage}{.5\textwidth}
        \imprimirpreambulo
    \end{minipage}%
    \vspace*{\fill}
   \end{center}
        
   Trabalho aprovado. \imprimirlocal, 24 de março de 2023:

   \assinatura{\textbf{\imprimirorientador} \\ Orientador} 
   \assinatura{\textbf{Professor} \\ Convidado 1}
   \assinatura{\textbf{Professor} \\ Convidado 2}
   %\assinatura{\textbf{Professor} \\ Convidado 3}
   %\assinatura{\textbf{Professor} \\ Convidado 4}
      
   \begin{center}
    \vspace*{0.5cm}
    {\large\imprimirlocal}
    \par
    {\large\imprimirdata}
    \vspace*{1cm}
  \end{center}
  
\end{folhadeaprovacao}
% ---

% ---
% Dedicatória
% ---
\begin{dedicatoria}
   \vspace*{\fill}
   \centering
   \noindent
   \textit{Este projeto é dedicado a todos os desenvolvedores que já falharam várias vezes, mas nunca desistiram de suas paixões e ideias.}

   //

   \begin{quote}
	\textit{``Herdei de meus ancestrais a tarefa de melhorar este mundo. E, para que possa entregá-lo a meus descendentes em melhor estado do que recebi, devo trabalhar, tanto quanto eles trabalharam ou ainda mais.'' - Bernard Shaw.}
   \end{quote}

   \vspace*{\fill}

\end{dedicatoria}
% ---

% ---
% Agradecimentos
% ---
\begin{agradecimentos}
Os agradecimentos principais são direcionados à Gerald Weber, Miguel Frasson,
Leslie H. Watter, Bruno Parente Lima, Flávio de Vasconcellos Corrêa, Otavio Real
Salvador, Renato Machnievscz\footnote{Os nomes dos integrantes do primeiro
projeto abn\TeX\ foram extraídos de
\url{http://codigolivre.org.br/projects/abntex/}} e todos aqueles que
contribuíram para que a produção de trabalhos acadêmicos conforme
as normas ABNT com \LaTeX\ fosse possível.

Agradecimentos especiais são direcionados ao Centro de Pesquisa em Arquitetura
da Informação\footnote{\url{http://www.cpai.unb.br/}} da Universidade de
Brasília (CPAI), ao grupo de usuários
\emph{latex-br}\footnote{\url{http://groups.google.com/group/latex-br}} e aos
novos voluntários do grupo
\emph{\abnTeX}\footnote{\url{http://groups.google.com/group/abntex2} e
\url{http://www.abntex.net.br/}}~que contribuíram e que ainda
contribuirão para a evolução do \abnTeX.

\end{agradecimentos}
% ---

% ---
% Epígrafe
% ---
\begin{epigrafe}
    \vspace*{\fill}
	\begin{flushright}
		\textit{``Não vos amoldeis às estruturas deste mundo, \\
		mas transformai-vos pela renovação da mente, \\
		a fim de distinguir qual é a vontade de Deus: \\
		o que é bom, o que Lhe é agradável, o que é perfeito.\\
		(Bíblia Sagrada, Romanos 12, 2)}
	\end{flushright}
\end{epigrafe}
% ---

% ---
% RESUMOS
% ---

% resumo em português
\setlength{\absparsep}{18pt} % ajusta o espaçamento dos parágrafos do resumo
\begin{resumo}
 Este trabalho apresenta um projeto de sistema de reserva e controle de
equipamentos audiovisuais para escola do Ensino Fundamental e Médio, visando
facilitar o controle de empréstimos de recursos. O projeto foi criado combase em
disciplinas como Engenharia de Software, projeto de Interface, Economia e
Mercado e Programação Orientada a Objetos. O sistema conta comuma interface
intuitiva que permite que os professores solicitem empréstimos de forma fácil,
enquanto os administradores gerenciam os empréstimos de maneiraeficiente.

 \textbf{Palavras-chave}:  Escola. Sistema. Reserva. Equipamentos.
\end{resumo}

% resumo em inglês
\begin{resumo}[Abstract]
 \begin{otherlanguage*}{english}
 This paper presents a project for a reservation and control system for
audiovisual equipment for Elementary and High Schools, with the aim of facilitating
the control of borrowing resources. The project was created basedon disciplines
such as Software Engineering, Interface design, Economy andMarket and Object
Oriented Programming. The system features an intuitive interface that allows
teachers to easily apply for grants, while administrators manage grants efficiently.

   \vspace{\onelineskip}
 
   \noindent 
   \textbf{Keywords}: School. System. Reservation. Equipment.
 \end{otherlanguage*}
\end{resumo}
% ---

% ---
% inserir lista de ilustrações
% ---
\pdfbookmark[0]{\listfigurename}{lof}
\listoffigures*
\cleardoublepage
% ---

% ---
% inserir lista de quadros
% ---
\pdfbookmark[0]{\listofquadrosname}{loq}
\listofquadros*
\cleardoublepage
% ---

% ---
% inserir lista de tabelas
% ---
\pdfbookmark[0]{\listtablename}{lot}
\listoftables*
\cleardoublepage
% ---

% ---
% inserir lista de abreviaturas e siglas
% ---
\begin{siglas}
  \item[ABNT] Associação Brasileira de Normas Técnicas
  \item[abnTeX] ABsurdas Normas para TeX (segundo desenvolvedores)
\end{siglas}
% ---

% ---
% inserir lista de símbolos
% ---
\begin{simbolos}
  \item[$ \Gamma $] Letra grega Gama
  \item[$ \Lambda $] Lambda
  \item[$ \zeta $] Letra grega minúscula zeta
  \item[$ \in $] Pertence
\end{simbolos}
% ---

% ---
% inserir o sumario
% ---
\pdfbookmark[0]{\contentsname}{toc}
\tableofcontents*
\cleardoublepage
% ---



% ----------------------------------------------------------
% ELEMENTOS TEXTUAIS
% ----------------------------------------------------------
\textual

% ----------------------------------------------------------
% Introdução (exemplo de capítulo sem numeração, mas presente no Sumário)
% ----------------------------------------------------------
\chapter{Introdução}
% ----------------------------------------------------------

O gerenciamento de empréstimos de biblioteca é uma atividade fundamental para as escolas e universidades, sendo que a utilização de sistemas eletrônicos tem sido cada vez mais comum. Com a evolução tecnológica, as opções de software de gerenciamento têm se expandido, permitindo que as instituições escolham a solução que melhor se adapte às suas necessidades.

Nesse contexto, o objetivo deste trabalho é apresentar uma solução tecnológica para o gerenciamento de empréstimos em uma escola fictícia, explorando suas principais funcionalidades e vantagens. Para tanto, serão realizadas pesquisas bibliográficas e desenvolvimento de um protótipo, buscando identificar as características da solução ideal para a instituição em questão.

Acredita-se que este trabalho poderá contribuir para a melhoria do gerenciamento de empréstimos de biblioteca, tornando o acesso aos materiais mais fácil para os professores e otimizando o trabalho dos profissionais envolvidos no processo de aprendizado.

Durante o período de estudos, foram abordados diversos conceitos importantes nas matérias de Economia e Mercado, Engenharia de Software, Programação Orientada a Objetos e Projeto de Interface com o Usuário. Em Economia e Mercado, aprendemos sobre o funcionamento das leis de oferta e demanda, os diferentes tipos de mercados e as políticas econômicas adotadas pelo governo. Já em Engenharia de Software, abordamos metodologias de desenvolvimento de software e técnicas de gerenciamento de projetos de software. Na disciplina de Programação Orientada a Objetos, aprendemos sobre os conceitos de classes, objetos, herança e polimorfismo. Por fim, na matéria de Projeto de Interface com o Usuário, estudamos sobre a importância da usabilidade e da experiência do usuário no desenvolvimento de interfaces gráficas para sistemas computacionais, bem como as técnicas e ferramentas utilizadas para a criação de interfaces intuitivas e eficientes. Todo esse conhecimento adquirido será fundamental para o desenvolvimento de projetos futuros na área de tecnologia da informação.

Postamos o protótipo final do trabalho em \url{https://www.stockedu.orbytesistemas.com/} \footnote{\url{http://www.latex-project.org/lppl.txt}}, o acesso para o resto da plataforma é "usuario1" e "senha1". Já o código fonte, assim como outros documentos importante de reuniões e afins, postamos no github, disponível em \url{https://github.com/Pedro-Laurenti/PIM-IV-UNIP}\footnote{\url{https://github.com/Pedro-Laurenti/PIM-IV-UNIP}}.

A Equipe

Pedro José Laurenti de Matos

% ----------------------------------------------------------
% PARTE
% ----------------------------------------------------------
\part{Pesquisa de mercado e metodologias}
% ----------------------------------------------------------

% ---
% Capitulo com exemplos de comandos inseridos de arquivo externo 
% ---
\include{abntex2-modelo-include-comandos}
% ---

\chapter{Conteúdos específicos do modelo de trabalho acadêmico}\label{cap_trabalho_academico}

\section{Quadros}

Este modelo vem com o ambiente \texttt{quadro} e impressão de Lista de quadros 
configurados por padrão. Verifique um exemplo de utilização:

\begin{quadro}[htb]
\caption{\label{quadro_exemplo}Exemplo de quadro}
\begin{tabular}{|c|c|c|c|}
	\hline
	\textbf{Pessoa} & \textbf{Idade} & \textbf{Peso} & \textbf{Altura} \\ \hline
	Marcos & 26    & 68   & 178    \\ \hline
	Ivone  & 22    & 57   & 162    \\ \hline
	...    & ...   & ...  & ...    \\ \hline
	Sueli  & 40    & 65   & 153    \\ \hline
\end{tabular}
\fonte{Autor.}
\end{quadro}

Este parágrafo apresenta como referenciar o quadro no texto, requisito
obrigatório da ABNT. 
Primeira opção, utilizando \texttt{autoref}: Ver o \autoref{quadro_exemplo}. 
Segunda opção, utilizando  \texttt{ref}: Ver o Quadro \ref{quadro_exemplo}.

% ----------------------------------------------------------
% PARTE
% ----------------------------------------------------------
\part{Referenciais teóricos}
% ----------------------------------------------------------

% ---
% Capitulo de revisão de literatura
% ---
\chapter{ ipsum dolLoremor sit amet}
% ---

% ---
\section{Aliquam vestibulum fringilla lorem}
% ---

\lipsum[1]

\lipsum[2-3]

% ----------------------------------------------------------
% PARTE
% ----------------------------------------------------------
\part{Resultados}
% ----------------------------------------------------------

% ---
% primeiro capitulo de Resultados
% ---
\chapter{Lectus lobortis condimentum}
% ---

% ---
\section{Vestibulum ante ipsum primis in faucibus orci luctus et ultrices
posuere cubilia Curae}
% ---

\lipsum[21-22]

% ---
% segundo capitulo de Resultados
% ---
\chapter{Nam sed tellus sit amet lectus urna ullamcorper tristique interdum
elementum}
% ---

% ---
\section{Pellentesque sit amet pede ac sem eleifend consectetuer}
% ---

\lipsum[24]

% ----------------------------------------------------------
% Finaliza a parte no bookmark do PDF
% para que se inicie o bookmark na raiz
% e adiciona espaço de parte no Sumário
% ----------------------------------------------------------
\phantompart

% ---
% Conclusão
% ---
\chapter{Conclusão}
% ---

\lipsum[31-33]

% ----------------------------------------------------------
% ELEMENTOS PÓS-TEXTUAIS
% ----------------------------------------------------------
\postextual
% ----------------------------------------------------------

% ----------------------------------------------------------
% Referências bibliográficas
% ----------------------------------------------------------
\bibliography{abntex2-modelo-references}

% ----------------------------------------------------------
% Glossário
% ----------------------------------------------------------
%
% Consulte o manual da classe abntex2 para orientações sobre o glossário.
%
%\glossary

% ----------------------------------------------------------
% Apêndices
% ----------------------------------------------------------

% ---
% Inicia os apêndices
% ---
\begin{apendicesenv}

% Imprime uma página indicando o início dos apêndices
\partapendices

% ----------------------------------------------------------
\chapter{Quisque libero justo}
% ----------------------------------------------------------

\lipsum[50]

% ----------------------------------------------------------
\chapter{Nullam elementum urna vel imperdiet sodales elit ipsum pharetra ligula
ac pretium ante justo a nulla curabitur tristique arcu eu metus}
% ----------------------------------------------------------
\lipsum[55-57]

\end{apendicesenv}
% ---


% ----------------------------------------------------------
% Anexos
% ----------------------------------------------------------

% ---
% Inicia os anexos
% ---
\begin{anexosenv}

% Imprime uma página indicando o início dos anexos
\partanexos

% ---
\chapter{Morbi ultrices rutrum lorem.}
% ---
\lipsum[30]

% ---
\chapter{Cras non urna sed feugiat cum sociis natoque penatibus et magnis dis
parturient montes nascetur ridiculus mus}
% ---

\lipsum[31]

% ---
\chapter{Fusce facilisis lacinia dui}
% ---

\lipsum[32]

\end{anexosenv}

%---------------------------------------------------------------------
% INDICE REMISSIVO
%---------------------------------------------------------------------
\phantompart
\printindex
%---------------------------------------------------------------------

\end{document}
